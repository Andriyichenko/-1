\section{\\単関数(Simple Function)}

\begin{dfn}[単関数\cite{kingman1966introduction}]
  単関数とは可測集合の指示関数の有限な線型結合のことである.
  より正確に述べれば,集合 $X$ 上の実数値単関数とは,
  $1_{A_i}$ を $A_i$ の指示関数として,$X$ の有限分割
  \[
    X \;=\; A_1 \sqcup A_2 \sqcup \dotsb \sqcup A_n
  \]
  と,適当な実数の定数 $\alpha_1, \dots, \alpha_n$ を取って
  \[
    f(x) \;=\; \sum_{i=1}^{n} \alpha_i\, 1_{A_i}(x)
  \]
  なる形に表すことのできる関数 
  \[
    f \colon X \;\longrightarrow\; \mathbb{R}
  \]
  を言う.単関数を可測空間 $(X,\Sigma)$ 上で考えるとき,
  この形の単関数が $\Sigma$–可測であるための必要十分条件は,
  任意の $A_i$ が $\Sigma$ に属することである.
  したがって可測関数のみを考える場合には,
  単関数を「互いに交わらない可測集合の有限列
  $A_1,\dots,A_n \in \Sigma$ で,
  それらの和が $X$ を被覆するもの」
  に関する指示関数の線型結合として定める。
  空間 $(X,\Sigma)$ 上に測度 $\mu$ が定義されるとき、単函数 $f$ の $\mu$ に関する積分は、
\[
  \int_X f \, d\mu \;:=\; \sum_{k=1}^{n} a_k \mu(A_k)
  \qquad\bigl(f = \sum_{k} a_k\,1_{A_k}\bigr)
\]
と定める。
\end{dfn}% the \\ insures the section title is centered below the phrase: AppendixA

\begin{lem}[ルベーグ積分との関係\cite{ito1963lebesgue}]
  どのような非負の可測関数 $f: X \to \mathbb{R}^{+}$ であっても、単調増加な非負の単関数の列の各点収束の極限として与えられる。実際、$f$ を測度空間 $(X, \Sigma, \mu)$ 上定義される、上述のような非負可測関数とする。各 $n \in \mathbb{N}$ に対し、$f$ の値域を、$2^{2n} + 1$ 個の区間で、その内の $2^{2n}$ 個が長さ $2^{-n}$ を持つようなものに区分する。すなわち、各 $n$ に対して、
  \[
  I_{n,k} = \left[ \frac{k-1}{2^n}, \frac{k}{2^n} \right) \quad \text{for } k = 1, 2, \ldots, 2^{2n}
  \]
  および
  \[
  I_{n, 2^{2n}+1} = [2^n, \infty)
  \]
  を定める(固定された $n$ に対して、各集合 $I_{n,k}$ は互いに素であり、実数直線の非負の部分を覆うことに注意されたい)。
  
  今、可測集合
  \[
  A_{n,k} = f^{-1}(I_{n,k}) \quad \text{for } k = 1, 2, \ldots, 2^{2n} + 1
  \]
  を定義する。このとき、単関数の増加列
  \[
  f_n = \sum_{k=1}^{2^{2n}+1} \frac{k-1}{2^n} \mathbf{1}_{A_{n,k}}
  \]
  は、$n \to \infty$ としたとき、$f$ へと各点収束する。$f$ が有界であるなら、その収束は一様収束である。
  
\end{lem}
